\chapter{Introduction}
%\end{center}

Risk modeling is the process of using statistical or mathematical techniques to estimate the
probability and potential impact of various risks, such as financial losses or natural disasters. It
allows organizations to identify and prioritize potential risks, and develop strategies to mitigate
or manage them, (Saunders \& Allen, 2007). In the financial industry, risk modeling is used to evaluate the
likelihood and impact of different events that could cause financial losses. This is done by
analyzing historical data and making assumptions about future scenarios. However, the
traditional methods of risk modeling, which rely on a limited set of historical data and
assumptions about future scenarios, may not be sufficient to capture the complexities and
uncertainties of the banking sector in Zimbabwe.\\ \\
In recent years, the availability of big data has provided new opportunities for risk modeling. Big
data, as defined by Gandomi et al. (2015), is a term used to describe datasets that are too large
and complex to be analyzed using traditional data processing techniques. Big data can be used to
analyze a much broader and more diverse set of information, which can provide a more
comprehensive understanding of the financial sector. This can help to identify new risks, or to
better understand the relationships between different factors that influence financial stability
(Luo, 2017).\\\\
The use of big data in risk modeling can also improve the accuracy and interpretability of the
models. Traditional risk models are often based on a limited set of assumptions and variables,
which can lead to oversimplification and bias. By incorporating more data, big data models can
capture more complex and nuanced relationships between variables. This can help to improve the
accuracy of the predictions and to identify the key drivers of financial instability (Li, 2018).
Automated monitoring systems are computer-based systems that are designed to automatically
collect, process, analyze and store data, and to generate alerts or notifications in case of specific
events or conditions (Chen et al., 2016). These systems can be used to improve the risk
management process in financial institutes by providing real-time monitoring, risk assessment,
compliance monitoring, automated reporting and artificial intelligence and machine learning
integration.\\\\
Real-time monitoring: Automated monitoring systems can be used to monitor financial
transactions in real-time, allowing financial institutes to detect any unusual or suspicious activity
immediately. This can help to prevent fraud and financial crime (Boudoukh et al., 2007)

\section{Background of the study}

Companies' production and organizational processes have undergone significant
transformations since the 1990s due to increased competition on a global scale and the
accessibility of both structured and unstructured information. As a result, there is a need to
enhance data analysis and traditional risk management systems, (Vasarhelyi et al., 2015). In order
to derive insights on information, it's important to take into account the organization of
knowledge by establishing archives and effectively managing and analyzing large volumes of
data that come in various formats and at high speeds. This led to the advancement of business
intelligence, which involved utilizing specialized software for processing information, using
new technologies with current systems, and enhancing infrastructure resources to amplify 
performance for data analysis, ultimately leading to the development of
Big Data analytics and the ability to gain new levels of knowledge.\\\\
The current digital landscape demands proficiency in diverse skill sets and programming
techniques to effectively analyze, scrutinize, cleanse, and model significant amounts of data
from a variety of sources, including the internet. Such data analysis is integral in making
informed decisions (Warren et al., 2015). In the realm of Big Data, the data derived from the
internet holds significant importance due to its vast potential for providing valuable insights,
particularly in the domain of predictive analysis. Key sources of operational data include accounting software,
 customer and employee management tools, and other related areas such as production management, purchasing and
  delivery applications, industrial and branch management, financial instruments, and risk assessment for banks.
   All of these sources are integrated to optimize their effectiveness.\\\\
Over the past few years, advancements in technology have paved the way for a substantial
transformation in the manner in which organizations share both internal and external
information, with a specific focus on areas such as governance, performance, and risk
management. This revolution in data sharing has laid the groundwork for a gradual evolution of
business and organizational processes, and has positioned itself as the cornerstone of what is
commonly referred to as the 'fourth industrial revolution' (Manyika et al., 2011). Big Data, which
comprises vast quantities of data that are heterogeneous, redundant, and unstructured, is
primarily evaluated and analyzed using risk-based logic that concentrates on credit, operational,
and compliance risks, such as those outlined in the General Data Protection Regulation
(GDPR) aimed at preventing money laundering. The significance of Big Data is
particularly apparent in the financial industries due to the strict
regulatory framework imposed by supervisory authorities such as the Reserve Bank Of
Zimbabwe and Insurance and Pensions Commission (IPEC) which enforces rigorous capital
regulations.\\\\
By adopting Big Data Analytics (BDA) and leveraging digital technologies to gather vast
quantities of data, businesses can align their risk management activities with their strategic
priorities, leading to the maximization of their overall value. This is possible due to the timely
reporting of sources of uncertainty, which allows the identification of anticipatory and proactive
actions that can be taken to enhance performance. The prioritization of risk knowledge and
measurement necessitates the delegation of specific responsibilities at all levels of the
organization, along with the establishment of an efficient reporting and communication system.
To achieve this, communication within the company (both bottom-up and top-down) is critical, as
is the dissemination of valid information to identify and catalog all outside risks to business
operations and make informed decision-making.\\\\
The presence of heightened uncertainties associated with political, regulatory, macroeconomic,
and technological factors has necessitated the adoption of a more integrated and continuous
approach towards risk management. This approach involves mitigating risky events and
ensuring widespread accountability throughout a company's organization. The effective
implementation of risk management policies and programs mandates the involvement of senior
executives to determine the fundamental principles that protect both internal and external
stakeholders (through social responsibility initiatives) while ensuring business continuity,
generating value over time, and adhering to behavioral ethics standards. It was inevitable to
reconsider organizational models and prioritize a systemic perspective of corporate risk
(Rasmussen, 1997; Floricel and Miller, 2001). Various standards have played a significant role
in formalizing the modern risk management approach. These standards provide principles and guidelines for incorporating the
risk management process into an organization's overall governance, from strategic planning to
reporting policies.


\section{Problem Statement}

Despite the growing importance of the banking Sector Zimbabwe, banks in the country continue
to face significant challenges when it comes to managing risk and predicting possible risks.
Traditional methods of risk management, based on historical data and statistical models, may not
be sufficient to identify and mitigate risks in a timely manner.\\\\
Big data and advancements in technology have the potential to provide more accurate and
efficient methods of identifying and mitigating risks in the financial industry. However, the use
of big data in risk modeling is still in its infancy in Zimbabwe, and there is a lack of research and
practical applications in this field.\\\\
Hence, the need of risk modeling with big data is to develop a more accurate and efficient way to
model risk using large datasets. Traditional risk modeling methods are limited by the amount of
data they can handle, which can result in inaccurate risk assessments and increased exposure to
losses. With the growth of big data, there is a need to develop new methods and techniques that
can effectively analyze and model risk using large, complex datasets.\\\\
\textbf{The thesis aims to address the following research questions:}
\begin{itemize}
    \item[$\bullet$] What are the current challenges in risk modeling with big data?
    \item[$\bullet$] What are the existing techniques for risk modeling with big data?
    \item[$\bullet$] How can machine learning algorithms be used to model risk with big data?
    \item[$\bullet$]What are the limitations of using big data for risk modeling, and how can they be addressed?
    \item[$\bullet$] What are the practical implications of using big data for risk modeling in the banking sector?\\
\end{itemize}
The research will focus on developing new approaches to risk modeling using big data, and
evaluating the performance of these methods in comparison to traditional approaches. The results
of this research can be used to improve risk management strategies and reduce losses for
organizations that rely on accurate risk assessments.

\section{Aims of the project}

To develop risk model using Machine learning and artificial intelligence algorithims using big data in Zimbabwe banking sector.

\section{Objectives of the study}
\begin{enumerate}
    \item To explore the potential of big data in risk modeling and identify the benefits and limitations of using big data for risk assessment.
    \item To develop and evaluate novel methods for risk modeling that leverage big data sources, such as real-time transactions, and other digital platforms.
    \item To investigate the impact of data quality and data preprocessing on the accuracy and reliability of risk models built with big data.
    \item To assess the ethical and privacy implications of using big data for risk modeling, and propose guidelines for responsible data use.
    \item To apply the developed risk models to real-world scenarios, such as financial risk management, fraud risk assessment,laundering risks, financial instability, or natural disaster prediction, and evaluate their effectiveness and practicality.
    \item To contribute to the existing literature on risk modeling with big data, by providing a comprehensive overview of the state of the art, identifying research gaps and proposing future directions for research.
\end{enumerate}

Overally, to advance our understanding of how big data can be used to improve risk assessment
and management, and to develop practical and effective methods for leveraging these data
sources.\par

\section{Significance of the study}

\begin{itemize}
    \item[$\bullet$] Improved risk management: Big data can provide more comprehensive and accurate risk
assessments, which can help companies better manage risk. By incorporating a larger set
of data, including real-time data, companies can have a more complete picture of the risks
they face and take steps to mitigate them.
    \item[$\bullet$] Better decision-making: Accurate risk models can help decision-makers make informed
decisions. By providing more accurate data, big data risk models can help
decision-makers understand the potential risks associated with a particular action or
decision.
    \item[$\bullet$] Enhanced competitiveness: By using big data risk models, companies can gain a
competitive advantage. By understanding risks more comprehensively and accurately,
companies can make better decisions and take actions that enable them to stay ahead of
their competitors.
    \item[$\bullet$] Innovation: The study of risk modeling with big data can lead to the development of new
risk assessment methodologies and models. This can drive innovation in risk
management and enable companies to better understand and manage risks in novel and
effective ways.\par 
\end{itemize}


\section{Project Layout}

The project comprises five chapters, starting with Chapter 1, which covers the introduction, background,
 problem statement, research objectives, and significance of the study.
 Chapter 2 presents the theoretical review related to the topic at hand, followed by Chapter 3, which outlines 
 the methodology used in the study. In Chapter 4, the data analysis and results are presented, and the study 
 concludes with Chapter 5, which provides a summary of the findings, conclusions, and recommendations