\chapter{Discussions, Conclusions and Recommendations}

\section{Discussions and Conclusions}

The aim of my undergraduate thesis was to develop a risk modeling approach using big data analytics to detect 
fraudulent transactions in the banking sector. Through the study, I collected a vast amount of transactional data 
from different sources, including customer transaction history, socio-demographic information, and other relevant 
variables that could help identify potential fraudulent transactions.\\\\
After conducting the research, the study concluded that risk modeling with big data is an effective approach to 
detecting fraudulent transactions in the banking sector. By utilizing advanced analytical techniques such as 
machine learning algorithms, the study was able to identify patterns and anomalies in transaction data that may 
indicate fraudulent behavior.\\\\
However, it is important to note that there may be limitations and challenges associated with big data analytics, 
such as data quality issues, privacy concerns, and model interpretability. As such, further research is needed to 
refine the risk modeling approach, address identified limitations, and improve the overall effectiveness of the 
approach.\\\\
The study has practical implications for the banking sector, as it can help institutions improve their fraud 
detection capabilities, reduce the risk of financial losses, and enhance their customer service. By identifying 
fraudulent transactions promptly, banks can take immediate action to protect their customers and maintain the 
integrity of their operations.\\\\
In addition to the conclusions mentioned earlier, the study also found that the use of big data analytics in the 
banking sector can provide several advantages over traditional methods of fraud detection. For example, big data 
analytics can analyze large volumes of transactional data in real-time, enabling banks to detect and respond to 
fraudulent activities more quickly and effectively.\\\\
Moreover, the study revealed that machine learning algorithms can be used to build predictive models that can 
accurately identify potential fraudulent transactions based on various features such as transaction frequency, 
amount, location, and time of day. These predictive models can be trained using historical data to continually 
improve the accuracy of fraud detection over time.\\\\
However, the study also highlighted that there are some challenges associated with implementing a risk modeling 
approach using big data analytics. For example, there may be issues with data quality and consistency that can 
affect the accuracy of the predictive models. Furthermore, the complexity of the algorithms used for risk modeling 
can make it difficult to interpret and understand the results, which may pose challenges for banks in terms of 
regulatory compliance.\\\\
Therefore, the study recommended that banks invest in technologies that can help to improve data quality, such as 
data cleansing and standardization tools. Furthermore, banks should also develop processes to ensure the 
interpretability and transparency of their risk modeling algorithms to comply with regulatory requirements.\\\\
In conclusion, the study has demonstrated that risk modeling with big data analytics is an effective approach to 
detecting fraudulent transactions in the banking sector. By utilizing advanced analytical techniques and predictive 
models, banks can identify fraudulent behavior accurately and more efficiently than traditional methods. However, 
to maximize the effectiveness of this approach, banks must address the challenges associated with data quality, 
algorithm complexity, and interpretability to ensure that the results are accurate, transparent, and compliant 
with regulatory requirements.

\section{Recommendations to the Banking Companies and law makers}

Recommendations for Banking Companies:

\begin{enumerate}
\item Implement risk modeling with big data analytics: Banks should adopt risk modeling with big data analytics as 
a core approach to detecting fraudulent transactions. By using advanced analytical techniques, banks can detect 
fraudulent behavior quickly and efficiently, reducing the risk of financial loss and protecting their customers.
\item Invest in data quality improvement: To maximize the effectiveness of risk modeling, banks must ensure that 
their data is of high quality, consistent, and reliable. Investing in data quality improvement tools such as data 
cleansing and standardization can help improve the accuracy of predictive models and reduce false positives.
\item Develop processes for algorithm interpretability: Banks must develop processes for algorithm interpretability 
to comply with regulatory requirements. This can be achieved by developing models that are transparent and 
explainable, and by providing clear and concise reports on the results of the models.
\end{enumerate}

Recommendations for Lawmakers:

\begin{enumerate}
    \item Promote big data's use in analytics in the financial institutes as banks etc: Lawmakers should encourage 
    the adoption of big data analytics in the banking sector to help detect and prevent fraudulent transactions. 
    This can be achieved by offering incentives such as tax breaks or grants to banks that invest in this 
    technology.
    \item Develop regulations to protect customer privacy: Lawmakers should develop regulations to protect customer 
    privacy and ensure that the use of big data analytics does not compromise the security of sensitive customer 
    data.
    \item Promote algorithmic transparency: Lawmakers should promote algorithmic transparency to ensure that the 
    results of risk modeling are accurate, transparent, and fair. This can be achieved by requiring banks to provide 
    clear and concise reports on the results of their models, and by encouraging them to develop models that are 
    transparent and explainable.\\
\end{enumerate}
In conclusion, the recommendations provided for banking companies and lawmakers can help promote the effective use 
of risk modeling with big data analytics to detect fraudulent transactions in the banking sector. By adopting these 
recommendations, banks can improve their fraud detection capabilities, reduce the risk of financial loss, and 
enhance their customer service. Likewise, lawmakers can help promote the adoption of this technology while 
protecting customer privacy and promoting algorithmic transparency.
